% ID: P004
% Title: Quadratic Residue Graph — Closed Form for f(n) and Infinite Collisions
% Tags: number theory, quadratic residues, CRT, combinatorics, graphs
% Source: ChatGPT (user-posed)

\documentclass[11pt]{article}
\usepackage{amsmath,amssymb}
\usepackage[a4paper, total={7in,10.2in}]{geometry}
\usepackage{parskip}
\setlength{\parskip}{1em}

\title{P004 — Quadratic Residue Graph Collisions}
\author{Chuah Jia Herng}
\date{February 2026}

\begin{document}
\maketitle

\section*{Problem}

    For a positive integer $n$, consider the graph with vertex set $\{0,1,\dots,n-1\}$. Two \emph{distinct} vertices $i\neq j$ are linked if and only if $i+j$ is a quadratic residue modulo $n$.

    Let $f(n)$ be the number of such unordered links.

    \begin{enumerate}
        \item Find a closed form for $f(n)$.
        \item Let $k$ be the greatest integer such that there exists $N$ with the following property: there exists some $n>N$ for which one can find distinct integers $n_1,\dots,n_k>N$ satisfying
        \[f(n)=f(n_1)=\cdots=f(n_k).\]
        Determine $k$.
    \end{enumerate}

\section*{Idea}

    Count links by grouping unordered pairs $\{i,j\}$ according to the residue class $s\equiv i+j\pmod n$. For each fixed $s$, the number of such pairs depends only on the parity of $n$ and (when $n$ is even) on the parity of $s$.

    Summing over those $s$ that are quadratic residues introduces the arithmetic function
    \[R(n)=\#\{x^2\bmod n:x\in\mathbb Z\},\]
    the number of quadratic residues modulo $n$, and in the even case also the number of odd quadratic residues.

    To study multiplicities of $f(n)$, construct explicit pairs of integers producing the same value and iterate the construction.

\section*{Solution}

\subsection*{Step 1: Counting pairs with a fixed sum}

    Fix $s\in\{0,1,\dots,n-1\}$ and define
    \[N_s=\#\bigl\{\{i,j\}:0\le i<j\le n-1,\ i+j\equiv s\pmod n\bigr\}.\]

    \textbf{Case 1: $n$ odd.}
    Since $2$ is invertible modulo $n$, the congruence $i\equiv s-i\pmod n$ has exactly one solution. Removing this diagonal solution leaves $n-1$ ordered solutions with $i\neq j$, corresponding to $(n-1)/2$ unordered pairs. Hence
    \[N_s=\frac{n-1}{2}\qquad(\forall s).\]

    \textbf{Case 2: $n$ even.}
    Write $n=2m$.
    \begin{itemize}
        \item If $s$ is odd, there is no diagonal solution, giving $m=n/2$ unordered pairs.
        \item If $s$ is even, there are two diagonal solutions, so removing them leaves $2m-2$ ordered solutions with $i\neq j$, giving $(n-2)/2$ unordered pairs.
    \end{itemize}

\subsection*{Step 2: Closed form for $f(n)$}

    Define
    \[R(n)=\#\{x^2\bmod n:x\in\mathbb Z\}.\]

    \textbf{If $n$ is odd,} each quadratic residue $s$ contributes $(n-1)/2$ unordered pairs, hence
    \[f(n)=\frac{n-1}{2}R(n).\]

    \textbf{If $n$ is even,} define $R_{\mathrm{odd}}(n)$ to be the number of odd quadratic residues modulo $n$. Summing contributions gives
    \[f(n)=\frac{n}{2}R_{\mathrm{odd}}(n)+\frac{n-2}{2}\bigl(R(n)-R_{\mathrm{odd}}(n)\bigr)=\frac{n-2}{2}R(n)+R_{\mathrm{odd}}(n).\]

\subsection*{Step 3: Infinite collision chains}

    \textbf{Claim.}
    If $u$ is odd and $3\nmid u$, then
    \[f(6u)=f(8u).\]

    \textbf{Proof.}
    Using multiplicativity and the values
    \[R(2)=2,\quad R(3)=2,\quad R(8)=3,\quad R_{\mathrm{odd}}(2)=1,\quad R_{\mathrm{odd}}(8)=1,\]
    we obtain
    \[R(6u)=4R(u),\quad R_{\mathrm{odd}}(6u)=R(u),\]
    \[R(8u)=3R(u),\quad R_{\mathrm{odd}}(8u)=R(u).\]
    Therefore
    \[f(6u)=\frac{6u-2}{2}\cdot4R(u)+R(u)=2(6u-1)R(u),\]
    \[f(8u)=\frac{8u-2}{2}\cdot3R(u)+R(u)=2(6u-1)R(u),\]
    proving the claim.

    Now define
    \[n_t=6\cdot4^t u\qquad(t=0,1,2,\dots).\]
    Then
    \[n_{t+1}=8\cdot4^t u,\]
    and applying the claim to $4^t u$ yields
    \[f(n_t)=f(n_{t+1})\quad\text{for all }t\ge0.\]

    Hence
    \[f(n_0)=f(n_1)=f(n_2)=\cdots,\]
    with all $n_t$ distinct and unbounded.

    Given any $N$ and $k$, choose $u$ large enough so that $n_0=6u>N$. Then $n_0,n_1,\dots,n_k>N$ and all have the same $f$-value. Thus no greatest $k$ exists.

\section*{Remarks}

    \begin{itemize}
        \item The formulas reduce the problem to the arithmetic of quadratic residues, governed by CRT and prime-power behavior.
        \item The collision construction exploits the identities $R(6u)=4R(u)$ and $R(8u)=3R(u)$ together with identical odd-residue counts.
    \end{itemize}

\end{document}
