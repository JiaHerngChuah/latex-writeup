% ID: P003
% Title: Digit-deletion invariant multiples
% Tags: number theory, modular arithmetic, decimal representation
% Source: China High School League 2025 Round 2 P3, https://artofproblemsolving.com/community/c6h3649034p35860447

\documentclass[11pt]{article}
\usepackage{amsmath,amssymb}
\usepackage[a4paper, total={7in,10.2in}]{geometry}
\usepackage{parskip}
\setlength{\parskip}{1em}

\title{P003 — Digit-deletion invariant multiples}
\author{Chuah Jia Herng}
\date{February 2026}

\begin{document}
\maketitle

\section*{Problem}

Find all positive integers $n$ such that the following holds: There exists a multiple $N$ of $n$, whose decimal representation contain all digits but $0$, and for each $i\in \{ 1,2,\ldots,9\}$, one can delete a digit $i$ from the decimal representation of $N$, so that the resulting number is again a multiple of $n$.

\section*{Idea}

Deleting a digit from a decimal number changes its value by a highly structured amount.
The condition that \emph{every} digit $1,\dots,9$ can be deleted while preserving divisibility by $n$
forces strong arithmetic constraints on $n$.

First, simple divisibility arguments exclude factors $3$ and $10$.
Conversely, when $n$ has neither of these factors, we explicitly construct a suitable $N$ by
combining modular control modulo $\gcd(n,10)=1$ with careful handling of powers of $2$ or $5$.

\section*{Solution}

\subsection*{Step 1: Necessary conditions}

\paragraph{Claim 1.} $10 \nmid n$.

\emph{Proof.}
If $10 \mid n$, then every multiple of $n$ ends in digit $0$.
This contradicts the requirement that $N$ contains no digit $0$.
\hfill$\square$

\paragraph{Claim 2.} $3 \nmid n$.

\emph{Proof.}
Assume $3 \mid n$. Then $3 \mid N$.
Since digit $1$ appears in $N$, delete one occurrence of it to obtain $N'$.
Divisibility by $3$ depends only on the sum of digits, and deleting a $1$ decreases the digit sum by $1$.
Thus $3 \nmid N'$, contradicting the assumption.
\hfill$\square$

Hence any admissible $n$ must satisfy
\[
10 \nmid n \quad\text{and}\quad 3 \nmid n.
\]

\subsection*{Step 2: Two lemmas}

\paragraph{Lemma 1.}
Let $d$ be a positive integer with $\gcd(d,10)=1$.
Then there exists an integer $k$ such that every residue class modulo $d$
is represented by some $k$-digit decimal number containing no digit $0$.

\emph{Proof.}
For each $k\ge 1$, let $S_k$ be the set of residues modulo $d$ represented by
$k$-digit numbers with no digit $0$.

Fix $k$ and choose, for each $r\in S_k$, a representative $x_r$.
Consider the two sets
\[
A=\{10^k+x_r : r\in S_k\}, \qquad
B=\{2\cdot 10^k+x_r : r\in S_k\}.
\]
All numbers in $A\cup B$ have $k+1$ digits and contain no digit $0$,
so their residues lie in $S_{k+1}$.

The maps $r\mapsto 10^k+x_r$ and $r\mapsto 2\cdot 10^k+x_r$ are injective modulo $d$.
If these two images had complete overlap modulo $d$, we would obtain
$10^k \equiv 0 \pmod d$, contradicting $\gcd(d,10)=1$.
Therefore,
\[
|S_{k+1}| \ge \min(d,\,2|S_k|).
\]

Since $|S_1|\ge 1$, repeated doubling implies that for sufficiently large $k$,
$|S_k|=d$. Thus every residue class modulo $d$ is represented by a $k$-digit
number with no digit $0$.
\hfill$\square$

\paragraph{Lemma 2.}
Let $p\in\{2,5\}$ and $\alpha\ge 1$.
There exists an $\alpha$-digit number containing no digit $0$
that is divisible by $p^\alpha$.

\emph{Proof.}
We proceed by induction on $\alpha$.

For $\alpha=1$, take $2$ or $5$.

Assume the claim holds for $\alpha=t$ and let $M$ be a $t$-digit no-zero number
with $p^t\mid M$.
Consider numbers of the form $B=10^t a+M$, obtained by prepending a digit $a$.
Then
\[
B = p^t\!\left(\left(\frac{10}{p}\right)^t a + \frac{M}{p^t}\right).
\]
Since $\gcd(10/p,p)=1$, the coefficient $(10/p)^t$ is invertible modulo $p$.
Thus there exists $a\in\{0,1,\dots,p-1\}$ such that $p$ divides the bracketed term.
If $a=0$, replace it by $p$, which is a valid nonzero digit.
Hence $B$ is divisible by $p^{t+1}$ and contains no digit $0$.
\hfill$\square$

\subsection*{Step 3: Sufficiency and construction}

Assume now that
\[
10 \nmid n \quad\text{and}\quad 3 \nmid n.
\]
Write
\[
n = p^\alpha d,
\]
where $p\in\{2,5\}$ (possibly $\alpha=0$) and $\gcd(d,10)=1$.

Choose $k>\alpha$ as in Lemma~1.
We construct
\[
N = A_1\,1\,A_2\,2\,\cdots\,A_9\,9\,B,
\]
where each $A_i$ is a $k$-digit no-zero block and $B$ is a $(k+\alpha)$-digit no-zero block.
This ensures that $N$ contains digits $1,\dots,9$ and no zeros.

Let $N_i$ denote the number obtained by deleting the displayed digit $i$.
A direct computation shows
\[
N-N_i = (9X_i+i)\,10^{\ell_i},
\]
where $X_i$ is the prefix before that digit and $\ell_i\ge \alpha$.

Using Lemma~1, choose the blocks $A_1,\dots,A_9$ inductively so that
\[
9X_i+i \equiv 0 \pmod d \quad (i=1,\dots,9).
\]
Then $d \mid (N-N_i)$ for all $i$.

Next, by Lemma~2, choose an $\alpha$-digit no-zero number $B_2$ divisible by $p^\alpha$,
and write $B=B_1B_2$ with $|B_1|=k$.
Since $\gcd(d,10)=1$, we may choose $B_1$ so that $d\mid N$.
Thus $n\mid N$.

Finally, since $\ell_i\ge \alpha$, we have $p^\alpha \mid 10^{\ell_i}$, so
$n\mid (N-N_i)$ for all $i$.
Therefore $n\mid N_i$ for every digit deletion, as required.

\section*{Remarks}

The conditions $3\nmid n$ and $10\nmid n$ are both necessary and sufficient.
The proof shows that the obstruction is purely arithmetic; once these two
divisibility constraints are removed, an explicit construction is always possible.

\end{document}
