% ID: P005
% Title: Multiples using a prescribed digit set (with all digits appearing)
% Tags: number theory, base representation, pigeonhole principle
% Source: original

\documentclass[11pt]{article}
\usepackage{amsmath,amssymb}
\usepackage[a4paper, total={7in,10.2in}]{geometry}
\usepackage{parskip}
\setlength{\parskip}{1em}

\title{P005 — Multiples using a prescribed digit set}
\author{Chuah Jia Herng}
\date{February 2026}

\begin{document}
\maketitle

\section*{Problem}

Let $b\ge 2$ be an integer base, and let
\[
0 \le u_1 < u_2 < \cdots < u_k \le b-1
\]
for some $k\ge 2$. Consider the digit set $D=\{u_1,u_2,\dots,u_k\}$.

Determine all $(u_1,\dots,u_k)$ with the following property:

\emph{For every positive integer $n$, there exists a positive multiple $N$ of $n$
whose base-$b$ representation uses only digits from $D$, and moreover each digit in $D$
appears at least once in the base-$b$ representation of $N$.}

\section*{Idea}

If $0\notin D$, then no number whose digits lie in $D$ can be divisible by $b$ (since a multiple
of $b$ must end in digit $0$ in base $b$). Hence $0\in D$ is necessary.

Assuming $0\in D$, we first solve the two-digit case $\{0,d\}$ using a classical repunit
pigeonhole argument (and appending zeros to handle factors of $b$). Then for a general digit set
$D=\{0,u_2,\dots,u_k\}$, we construct for each $u_i$ a multiple of $n$ using only digits $\{0,u_i\}$,
and concatenate these blocks in base $b$. Concatenation preserves divisibility and guarantees every digit appears.

\section*{Solution}

\subsection*{Step 1: Necessity of $u_1=0$}

Assume the stated property holds. Apply it to $n=b$.
Any multiple of $b$ written in base $b$ must end with digit $0$, hence its base-$b$ representation
must contain the digit $0$. Therefore $0\in D$, i.e.\ $u_1=0$.

Thus $u_1=0$ is necessary.

\subsection*{Step 2: A two-digit lemma}

\paragraph{Lemma.}
Let $b\ge2$, let $d\in\{1,2,\dots,b-1\}$, and let $n\ge1$.
Then there exists a positive multiple $M$ of $n$ whose base-$b$ representation consists only of digits
$0$ and $d$. Moreover, one can choose $M$ so that both digits $0$ and $d$ appear at least once.

\emph{Proof.}
Define the base-$b$ repunits
\[
R_k = 1+b+\cdots+b^{k-1}\qquad (k\ge1),
\]
whose base $b$ representation is the string $11\cdots 1$ of length $k$.

First suppose $\gcd(n,b)=1$.
Consider the $n+1$ residues modulo $n$:
\[
R_0:=0,\ R_1,\ R_2,\dots,\ R_n.
\]
By the pigeonhole principle, there exist $0\le i<j\le n$ such that
$R_i\equiv R_j\pmod n$.
Then
\[
0\equiv R_j-R_i=b^iR_{j-i}\pmod n.
\]
Since $\gcd(n,b)=1$, we may cancel $b^i$, obtaining $R_{j-i}\equiv 0\pmod n$.
Thus $n\mid R_k$ for some $k$, hence $n\mid dR_k$, and $dR_k$ uses only digit $d$ in base $b$.

For a general $n$, write $n=n_0 n_1$ where $n_0$ is the largest divisor of $n$ whose prime factors
all divide $b$, so that $\gcd(n_1,b)=1$.
Choose $t\ge0$ with $n_0\mid b^t$, and choose $k$ with $n_1\mid R_k$ by the coprime case.
Then
\[
M_0 := b^t\cdot dR_k
\]
satisfies $n\mid M_0$, and its base-$b$ digits are only $0$ and $d$.

Finally, if $M_0$ has no digit $0$ (this happens only when $t=0$), replace it by $M:=bM_0$.
Then $n\mid M$ and the base-$b$ representation appends a trailing zero, so both digits $0$ and $d$
appear. \hfill$\square$

\subsection*{Step 3: Sufficiency when $u_1=0$ and $k\ge2$}

Assume $u_1=0$ and $k\ge2$. Fix any $n\ge1$.

For each $i=2,3,\dots,k$, apply the lemma with $d=u_i$ to obtain a positive multiple $M_i$ of $n$
whose base-$b$ digits are only $0$ and $u_i$, and such that both digits $0$ and $u_i$ occur.
Let $\ell_i$ be the number of base-$b$ digits of $M_i$.

Define
\[
N := M_2\,b^{\ell_3+\cdots+\ell_k} + M_3\,b^{\ell_4+\cdots+\ell_k} + \cdots + M_{k-1}b^{\ell_k} + M_k.
\]

Each term is divisible by $n$, hence $n\mid N$.
In base $b$, multiplying by $b^m$ shifts the representation left by $m$ digits (appending $m$ zeros),
so the base-$b$ representation of $N$ is the concatenation of the representations of
$M_2,M_3,\dots,M_k$.
Therefore $N$ uses only digits from $D$, and for each $i\ge2$, the digit $u_i$ appears in $M_i$.
Also, digit $0$ appears in every $M_i$, hence appears in $N$.

Thus $N$ is a positive multiple of $n$, uses only digits from $D$, and contains every digit of $D$
at least once.

\section*{Conclusion}

The stated property holds if and only if $u_1=0$ and $k\ge2$.

\[
\boxed{\text{All solutions are }(u_1,\dots,u_k)\text{ with }u_1=0\text{ and }k\ge2.}
\]

\end{document}
