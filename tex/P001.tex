% ID: P001
% Title: Continuous solutions to f(x) = (x+1)f(x^2)
% Tags: functional equations, continuity
% Source: SMMC 2024 B2, https://artofproblemsolving.com/community/c6h3418646p32922452

\documentclass[11pt]{article}
\usepackage{amsmath,amssymb}
\usepackage[a4paper, total={7in,10.2in}]{geometry}
\usepackage{parskip}
\setlength{\parskip}{1em}

\title{P001 — SMMC 2024 B2}
\author{Chuah Jia Herng}
\date{February 2026}

\begin{document}
\maketitle

\section*{Problem}
Determine all continuous functions $f : \mathbb{R}\setminus\{1\}\to\mathbb{R}$ such that
\[
f(x)=(x+1)f(x^2)
\]
for all $x\in \mathbb{R}\setminus\{-1,1\}$.

\section*{Idea}
Introduce $g(x)=(x-1)f(x)$ to remove the factor $(x+1)$.
Then the equation becomes the self-similarity $g(x)=g(x^2)$.
Iterating squaring sends $|x|<1$ to $0$, while iterating square-roots sends $x>1$ to $1^+$.
Continuity at $-1$ forces a right-limit at $1^+$, which pins down $g$ for all $x>1$.

\section*{Solution}
Define $g(x)=(x-1)f(x)$ for $x\neq 1$. Since $f$ is continuous on $\mathbb{R}\setminus\{1\}$,
so is $g$. For $x\neq \pm 1$,
\[
g(x)=(x-1)f(x)=(x-1)(x+1)f(x^2)=(x^2-1)f(x^2)=g(x^2).
\tag{1}
\]

\textbf{Step 1: $g$ is constant on $(-1,1)$.}
Fix $x\in(-1,1)$. Then $x^{2^n}\to 0$. By iterating (1),
\[
g(x)=g(x^{2^n})\quad\text{for all }n.
\]
Letting $n\to\infty$ and using continuity at $0$ gives $g(x)=g(0)$ for all $x\in(-1,1)$.
Write $A:=g(0)$. By continuity at $-1$ (note $-1\in\mathbb{R}\setminus\{1\}$), we have $g(-1)=A$.

\textbf{Step 2: the right-limit at $1$ equals $A$.}
For $t>1$, set $x=-\sqrt t< -1$. Then $x^2=t$ and (1) gives $g(t)=g(x)$.
Let $t\to 1^+$, so $x\to -1^-$. Continuity at $-1$ yields
\[
\lim_{t\to 1^+} g(t)=\lim_{x\to -1^-} g(x)=g(-1)=A.
\tag{2}
\]

\textbf{Step 3: $g$ is constant on $(1,\infty)$.}
Fix $y>1$ and define $y_n:=y^{1/2^n}$, so $y_n\to 1^+$. From (1),
\[
g(y)=g(y_n)\quad\text{for all }n.
\]
Letting $n\to\infty$ and using (2) gives $g(y)=A$. Hence $g(x)\equiv A$ for all $x>1$.

\textbf{Step 4: $g$ is constant on $(\infty,\infty)$.}
From $g(x) = g(x^2) = g(-x)$, since $g(x) = A$ for $x \ge 0$, $g(x) = A$ for $x \le 0$, i.e. $g(x)\equiv A$ for all $x\neq 1$.

\textbf{Step 5: Describing $f$.}
\[
(x-1)f(x)=A \quad\Rightarrow\quad f(x)=\frac{A}{x-1}\qquad (x\neq 1).
\]
This function is continuous on $\mathbb{R}\setminus\{1\}$ and satisfies the given equation.

\section*{Answer}
\[
\boxed{f(x)=\frac{c}{x-1}\ \text{for some constant }c\in\mathbb{R}.}
\]

\end{document}
