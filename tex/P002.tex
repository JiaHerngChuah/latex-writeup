% ID: P002
% Title: Iterating the square of digit sum on 2^1990
% Tags: number theory, digit sums, dynamical systems
% Source: IMO SL 1990 Problem 8, https://artofproblemsolving.com/community/c6h220947p1225217

\documentclass[11pt]{article}
\usepackage{amsmath,amssymb}
\usepackage[a4paper, total={7in,10.2in}]{geometry}
\usepackage{parskip}
\setlength{\parskip}{1em}

\title{P002 — IMO SL 1990 Problem 8}
\author{Chuah Jia Herng}
\date{February 2026}

\begin{document}
\maketitle

\section*{Problem}
For a given positive integer $k$ denote the square of the sum of its digits by $f_1(k)$ and let
$f_{n+1}(k)=f_1(f_n(k))$. Determine the value of $f_{1991}(2^{1990})$.

\section*{Idea}
Let $s(n)$ be the sum of decimal digits of $n$, so $f_1(n)=s(n)^2$.
Since $s(n)\equiv n\pmod 9$, we get $f_{m+1}(k)\equiv f_m(k)^2\pmod 9$.
For $k=2^{1990}$ this forces the residues modulo $9$ to alternate between $4$ and $7$.
Separately, repeated digit-sum squaring rapidly drives the values below $100$,
forcing entry into the $2$-cycle $169\leftrightarrow 256$; the parity of $m$ then determines the answer.

\section*{Solution}

\subsection*{Step 1: A modulo $9$ recurrence}
Let $s(n)$ denote the sum of decimal digits of $n$. Then $f_1(n)=s(n)^2$.
Since $s(n)\equiv n\pmod 9$ for all integers $n$, we have
\[
f_1(x)=s(x)^2\equiv x^2 \pmod 9.
\]
Hence for all $m\ge 1$,
\[
f_{m+1}(k)=f_1(f_m(k))\equiv f_m(k)^2 \pmod 9.
\tag{1}
\]

Now $2^6\equiv 1\pmod 9$, and $1990\equiv 4\pmod 6$, so
\[
2^{1990}\equiv 2^4=16\equiv 7\pmod 9.
\]
Therefore
\[
f_1(2^{1990})\equiv (2^{1990})^2 \equiv 7^2\equiv 4\pmod 9.
\]
Using (1), the residues evolve by squaring:
\[
4\mapsto 4^2\equiv 7,\qquad 7\mapsto 7^2\equiv 4.
\]
Thus
\[
f_m(2^{1990})\equiv
\begin{cases}
4 \pmod 9,& m\text{ odd},\\
7 \pmod 9,& m\text{ even}.
\end{cases}
\tag{2}
\]

\subsection*{Step 2: Rapid descent to small values}
The number $2^{1990}$ has $599$ digits, hence
\[
s(2^{1990})\le 9\cdot 599=5391,
\]
and therefore
\[
f_1(2^{1990})=s(2^{1990})^2\le 5391^2<3\times 10^7.
\]
Thus $f_1(2^{1990})$ has at most $8$ digits, giving
\[
s(f_1(2^{1990}))\le 72 \quad\Rightarrow\quad
f_2(2^{1990})\le 72^2=5184.
\]
Continuing similarly,
\[
f_3(2^{1990})\le 36^2=1296,\qquad
f_4(2^{1990})\le 18^2=324,\qquad
f_5(2^{1990})\le 9^2=81.
\]

Hence $f_5(2^{1990})$ is a perfect square not exceeding $81$.
By (2), $f_5(2^{1990})\equiv 4\pmod 9$, so
\[
f_5(2^{1990})\in\{4,49\}.
\]

\subsection*{Step 3: Entry into a $2$-cycle}
The subsequent values are forced:
\[
f_1(4)=16,\quad f_1(16)=49,\quad f_1(49)=169,\quad
f_1(169)=256,\quad f_1(256)=169.
\]
Thus the sequence eventually enters the $2$-cycle
\[
169 \leftrightarrow 256.
\]

Moreover,
\[
169\equiv 7\pmod 9,\qquad 256\equiv 4\pmod 9,
\]
so by (2) we must have $f_m(2^{1990})=256$ for odd $m$ and $169$ for even $m$
once the cycle is reached.

Since $1991$ is odd,
\[
f_{1991}(2^{1990})=256.
\]


\section*{Answer}
\[
\boxed{256}.
\]

\end{document}
